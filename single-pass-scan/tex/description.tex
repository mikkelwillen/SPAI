%problem presentation
The goal of this project is to implement a robust single-pass-scan in CUDA.
Scans are very important, but slow implementations of scans causes bottlenecks.
The common implementation makes $O(4n)$ memory operations, whereas the single-pass-scan should be able to do the work with just $O(2n)$ memory operations.
For comparison, the reduce-then-scan approach utilizes $O(3n)$ memory operations.

For this project we try to implement the single-pass-scan algorithm in CUDA.
We want to start with an implementation more similar to the chained scan, and then alter this to be the single-pass-scan, thus comparing the runtimes.

In \autoref{sec:algorithm} we describe the single-pass-scan algorithm as presented in the paper \cite{spsArticle}.
In \autoref{sec:implementation} we present our implementation of the single-pass-scan, and the considerations that we had along the way.
In \autoref{sec:performance} we present the performance of our implementation(s), and lastly we conclude in \autoref{sec:conclusion}.

Enjoy!
